

\chapter{Descrição Geral}

\paragraph{}

\section{Perspectiva}
Este trabalho irá resultar numa aplicação que possibilite aos intervenientes poderem efectuar operações sobre os momentos de avaliação da ESTIG num repositório comum que mantenha os respectivos intervenientes informados das eventuais alterações aos registos associados ao seu perfil.\\

\subsection{Interfaces}
\subsubsection{Sistema}
O sistema deverá funcionar como repositório para todos os momentos de avaliação dos cursos da Estig.

\subsubsection{Utilizador}
A interface de utilizador deverá ser compatível com os principais browsers e respeitar as principais regras de usabilidade e acessibilidade.

\subsubsection{Software}
O projecto será implementado na linguagem PHP com ligação a base de dados mySQL. Serão utilizados os API\'s jQuery e Smarty.

\subsection{Funções}
O sistema implementado irá suportar as seguintes funções:

\begin{list}{-}{}
\item \textbf{Efectuar Login}
\item \textbf{Visualizar Momentos de Avaliação}
\item \textbf{Validar / Cancelar Validação de Momentos de Avaliação}
\item \textbf{Inscrever / Cancelar Inscrições em Momentos de Avaliação}
\item \textbf{Marcar / Cancelar Marcação de Novos Momentos de Avaliação}
\end{list}

\subsection{Características do Utilizador}
Os utilizadores da aplicação serão alunos, docentes e coordenadores dos cursos leccionados na ESTIG. A sua utilização esporádica e em periodos de tempo espaçados faz com que em termos de usabilidade se tenha especial atenção à facilidade de aprendizagem no uso das interfaces.