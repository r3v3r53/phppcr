\section{VALIDAR AVALIAÇÕES}

\subsection{Descrição}
O Coordenador de Curso pode validar, cancelar ou alterar as datas das avaliações das disciplinas do curso a ele associado 	

\subsection{Pré-condições}
O coordenador deverá estar logado e apenas terá acesso a avaliações de disciplinas do seu curso.

\subsection{Situações de Falha}
1. O Coordenador tenta validar uma avaliação já validada ou cancelar a validação de uma por validar.

\subsection{Actores}
Coordenador

\subsection{Cenário Principal}

\subsubsection{Validar Avaliação} 
\prettyref{fig:validar_avaliacao}
O coordenador:

1. Ao visualizar uma disciplina, escolhe validar.

2. O sistema regista a validação.

3. Caso a operação corra com sucesso:

3.1. O sistema obtem a lista de docentes e alunos da disciplina.

3.2. O sistema consulta o endereço de cada um dos docentes e envia um email com os detalhes da operação.

3.3. O sistema actualiza o calendario,

3.4. O sistema apresenta a mensagem de operação efectuada com sucesso ao coordenador.

3.5 O sistema apresenta o calendario

4. Caso ocorra algum erro, o mesmo é apresentado ao coordenador

\subsection{Extensões ou Variações} 
\subsubsection{Cancelar Validação}
 \prettyref{fig:cancelar_validacao}

O Coordenador:

1. Ao visualizar uma disciplina, escolhe cancelar a sua validação.

2. O sistema assinala a avaliação por validar

3. Caso a operação corra com sucesso:

3.1. O sistema obtem a lista de docentes e alunos da disciplina.

3.2. O sistema consulta o endereço de cada um dos docentes e envia um email com os detalhes da operação.

3.3. O sistema actualiza o calendário

3.4. O sistema apresenta a mensagem de operação efectuada com sucesso ao coordenador.

3.5 O sistema apresenta o calendário

4. Caso ocorra algum erro, o mesmo é apresentado ao coordenador.

\subsubsection{Alterar data de Momento de Avaliação}
 \prettyref{fig:alterar_data}

O Coordenador:

1. Ao visualizar uma avaliação, escolhe alterar a sua data.

2. O sistema apresenta o formulário para alterar a data da avaliação

3. O coordenador efectua a alteração

4. Caso a operação corra com sucesso:

4.1. O sistema obtem a lista de docentes e alunos da disciplina.

4.2. O sistema consulta o endereço de cada um dos docentes e envia um email com os detalhes da operação.

4.3. O sistema actualiza o calendário

4.4. O sistema apresenta a mensagem de operação efectuada com sucesso ao coordenador.

4.5 O sistema apresenta o calendário

5. Caso ocorra algum erro, o mesmo é apresentado ao coordenador.