\chapter{Interface Java}

Foi iniciada uma pequena implementação de uma aplicação em Java para interagir com a base de dados criada no projecto.\\

Demonstramos aqui algumas funções da mesma, apresentando printscreens e a parte mais relevante do código criado para a sua execução.

\section{Login}
Efectua o login no sistema utilizando um username e password fornecidos pelo utilizador.
De seguida obtém o nome do utilizador logado para incluir na mensagem no interface. 

\begin{figure}[!htbp]
\centering
\includegraphics{depoislogin.jpg}
\caption{Interface: Login}
\label{fig:interface1}
\end{figure}

\subsection{Código}
\begin{lstlisting}
public void connect(String host, String user, String pass) {
		try
		{
			String url = "jdbc:sqlserver://" + host;
			url += ";databaseName=BD2;";
			url += "username="+user;
			url += ";password="+pass;
			Class.forName("com.microsoft.sqlserver.jdbc.SQLServerDriver");
			this.conn = DriverManager.getConnection(url);
			if(this.getNome().equals("")){
				this.resultados.setText("Logado como Administrador");
			} else {
				this.resultados.setText(this.getNome());
			}
			//this.resultados.setText(this.conn.toString());
			
		} catch (ClassNotFoundException | SQLException ex) {
			this.resultados.setText(ex.getMessage());
		}
	}

public String getNome(){
		Statement statement = null;
		ResultSet results 	= null;
		String nome = new String();
		try{
			statement = this.conn.createStatement();
			results = statement.executeQuery(
					"EXECUTE p_get_dados"
			);
			results.next();
			nome = "Logado como " + results.getNString("nome");
			System.out.println(results.getNString("nome"));
		} catch(SQLException ex){
			System.out.println(ex.getMessage());
		} 
		return nome;
	}
\end{lstlisting}

\section{Consultar dados de Utilizador}
Esta funcionalidade apenas funciona com um utilizador do role sacad, pois é a única que pode efectuar um SELECT na tabela utilizadores bem como executar a stored procedure p\_get\_user\_data

\begin{figure}[!htbp]
\centering
\includegraphics{verdadosutilizador.jpg}
\caption{Interface: Ver os dados de um utilizador}
\label{fig:dados1}
\end{figure}

\subsection{Código}
\begin{lstlisting}
public void actionPerformed(ActionEvent e)
    {
	ResultSet users = 
			this.bd.getRows(
					"SELECT username FROM utilizador ORDER BY username ASC;");
		
		try {
			while(users.next()){
				usernames.add(users.getString("username"));
			}
		} catch (SQLException e) {
			this.bd.resultados.setText(e.getMessage());
		}

	ResultSet dads = 
				this.bd.getRows(
						"execute p_get_user_data " + this.lista.getSelectedValue() );
		try {
			dads.next();
			this.dados.setText("");
			this.dados.append("Nome: " + dads.getString("nome") +"\n");
			this.dados.append("Password: " + dads.getString("pass") +"\n");
			this.dados.append("Email: " + dads.getString("email") +"\n");
			this.dados.append("Morada: " + dads.getString("morada") +"\n");
			this.dados.append("Telefone: " + dads.getString("telefone") +"\n");
			this.dados.append("Roles: ");
			dads = this.bd.getRows(
					"execute p_get_user_roles " + this.lista.getSelectedValue());
			try{
				while(dads.next()){
					this.dados.append(" ["+dads.getString("role")+"] " );
				}
			} catch(SQLException e2){
				this.bd.resultados.setText(e2.getMessage());
			}
		} catch (SQLException e1) {
			this.bd.resultados.setText(e1.getMessage());
		}
	}
public ResultSet getRows(String query){
		ResultSet results = null;
		try{
			this.statement = this.conn.createStatement();
			results = statement.executeQuery(query);
		} catch (SQLException ex) {
			this.resultados.setText(ex.getMessage());
		}
		return results;
	}
\end{lstlisting}

\section{Criar um novo Curso / Ano}

\begin{figure}[!htbp]
\centering
\includegraphics{cursoano.jpg}
\caption{Interface: Criar (associar) um novo Curso/ano}
\label{fig:curso1}
\end{figure}

\subsection{Código}
\begin{lstlisting}
private void utilizadores(){
		ArrayList<String> usernames = new ArrayList<String>();
		String[] itens;
		ResultSet users = 
			this.bd.getRows(
					"SELECT num_utilizador, username FROM utilizador ORDER BY username ASC;");
		
		try {
			while(users.next()){
				System.out.println(users.getString("num_utilizador") + ", " + users.getString("username"));
				usernames.add(users.getString("username"));
				this.utilizadores.put(users.getString("username"), users.getInt("num_utilizador"));
			}
		} catch (SQLException e) {
			this.bd.resultados.setText(e.getMessage());
		}
		itens = usernames.toArray(new String[usernames.size()]);
		this.utilizador =  new JList<String>(itens);
		GridLayout box3 = new GridLayout(2,1);
		JPanel ttres = new JPanel(box3);
		ttres.add(this.tres);
		ttres.add(this.utilizador);
		
		this.add(ttres, BorderLayout.EAST);
	}
	
	private void curso(){
		ArrayList<String> usernames = new ArrayList<String>();
		String[] itens;
		ResultSet users = 
			this.bd.getRows(
					"SELECT num_curso, nome_curso FROM curso ORDER BY nome_curso ASC;");
		try {
			while(users.next()){
				usernames.add(users.getString("nome_curso"));
				this.cursos.put(users.getString("nome_curso"), users.getInt("num_curso"));
			}
		} catch (SQLException e) {
			this.bd.resultados.setText(e.getMessage());
		}
		itens = usernames.toArray(new String[usernames.size()]);
		
		this.curso =  new JList<String>(itens);
		
		GridLayout box2 = new GridLayout(2,1);
		JPanel tdois = new JPanel(box2);
		tdois.add(this.dois);
		tdois.add(this.curso);
		
		this.add(tdois, BorderLayout.CENTER);
	}
	
	private void ano(){
		ArrayList<String> usernames = new ArrayList<String>();
		String[] itens;
		ResultSet users = 
			this.bd.getRows(
					"SELECT num_ano, titulo_ano FROM ano_lectivo ORDER BY titulo_ano ASC;");
		try {
			while(users.next()){
				usernames.add(users.getString("titulo_ano"));
				this.anos.put(users.getString("titulo_ano"), users.getInt("num_ano"));
			}
		} catch (SQLException e) {
			this.bd.resultados.setText(e.getMessage());
		}
		itens = usernames.toArray(new String[usernames.size()]);
		this.ano =  new JList<String>(itens);
		
		GridLayout box1 = new GridLayout(2,1);
		JPanel tum = new JPanel(box1);
		tum.add(this.um);
		tum.add(this.ano);
		
		this.add(tum, BorderLayout.WEST);
		
	}
	
	public void actionPerformed(ActionEvent e){
		String query = "INSERT INTO curso_ano (num_curso, num_ano, num_coordenador) " 
				+ " VALUES('"+this.cursos.get(this.curso.getSelectedValue())+"', '"+ 
				this.anos.get(this.ano.getSelectedValue()) + "', '"+
				this.utilizadores.get(this.utilizador.getSelectedValue())+"');";
		try {
			this.bd.statement = this.bd.conn.createStatement();
			this.bd.statement.executeQuery(query);
			this.bd.resultados.setText("Curso/Ano Criado");
		} catch (SQLException e1) {
			this.bd.resultados.setText(e1.getMessage());
		}
    }
\end{lstlisting}