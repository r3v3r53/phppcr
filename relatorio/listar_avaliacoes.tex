\section{LISTAR AVALIAÇÕES}

\subsection{Descrição}
A cada acesso ao sistema, o mesmo consulta as avaliações associadas ao utilizador para semre apresentadas no calendário no intervalo de datas definido, bem como os detalhes das avaliações do dia seleccionado.

\subsection{Pré-condições}
O utilizador deverá estar logado e apenas terá acesso a avaliações a ele associadas como student, teacher ou coordinator.

\subsection{Actores}
User

\subsection{Cenário Principal}

\subsubsection{Listar avaliações entre datas} 
\prettyref{fig:listar_entre_datas}
O user:

1. Acede ao sistema, já logado.

2. O sistema obtem a lista de cursos a ele associados, passando pelos mesmo passos já descritos no login.

3. O sistema obtem o intervalo de datas definido.

4. O sistema obtem a lista de avaliações definidas no intervalo de datas.

5. É apresentado ao utilizador a lista de avaliações no calendário.

\subsection{Extensões ou Variações} 
\subsubsection{Listar detalhes de avaliações}
 \prettyref{fig:listar_detalhes}

O aluno:

1. Ao visualizar uma disciplina, escolhe cancelar a inscrição.

2. O sistema remove o registo na tabela AVALAÇÃO\_ALUNO.

3. O sistema obtem a lista de docentes da disciplina.

4. O sistema consulta o endereço de cada um dos docentes e envia um email com os detalhes da operação.

5. O sistema apresenta a mensagem de operação efectuada com sucesso ao aluno.

6. O sistema gera novamente a lista de avaliações para o intervalo de tempo actual.

7. É apresentado o painel de informações da avaliação e o calendário. 