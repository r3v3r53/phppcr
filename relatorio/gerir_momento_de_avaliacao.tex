\section{GERIR MOMENTO DE AVALIAÇÃO}
\subsection{Descrição} 
O docente pode criar, alterar e cancelar momentos de avaliação para as disciplinas em que seja docente.

\subsection{Pré-condições}
Apenas é possível criar momentos de avaliação para as disciplinas em que o docente esteja associado.

\subsection{Actores} 
Docente 

\subsection{Cenário Principal}
 \prettyref{fig:novo_momento_de_avaliacao} 
 \subsubsection{Novo Momento de Avaliação}
O Docente:

1. Escolhe criar um novo momento de avaliação.

2. É apresentado pelo sistema o formulário para inserir os dados

3. Em caso da operação correr com sucesso:

3.1 O sistema consulta os endereços de email do coordenador e dos restantes docentes da disciplina

3.2 É enviado uma mensagem para cada endereço a notificar a criação da avaliação

2.4 O sistema apresenta uma mensagem a informar do sucesso da operação

2.5 O sistema actualiza as informações do calendário

2.6 É apresentado novamente o calendário ao docente

4. Caso o aconteça um erro é apresentada a mensagem de erro ao docente.\\ 

\subsection{Extensões ou Variações} 
 \prettyref{fig:alterar_momento_de_avaliacao}
\subsubsection{Alterar Momento de Avaliação}

O Docente:

1. Escolhe Alterar um momento de Avaliação.

2. O sistema apresenta o formulário para as alterações.

3. O utilizador preenche o formulário.

4. O sistema regista as alterações à avaliação.

5. Caso a operação seja efectuada com sucesso:

5.1 O sistema consulta os endereços de email do coordenador do curso, restantes docentes e alunos da disciplina

5.2. O sistema envia uma mensagem para os endereços de email

5.3 O sistema apresenta uma mensagem a dar conta do sucesso da operação

6. Caso ocorra algum erro é apresentado o erro ao utilizador

\subsubsection{Cancelar Momento de Avaliação}
 \prettyref{fig:cancelar_momento_de_avaliacao}

O Docente:

1. Escolhe cancelar um momento de avaliação.

2. O sistema procede ao seu cancelamento.

3. Caso a operação ocorra com sucesso:

3.1. O sistema consulta os endereços de email do coordenador, dos restantes docentes e alunos da disciplina

3.2. O sistema envia uma mensagem para os endereços de email a dar conta do cancelamento

3.3 O sistema actualiza o calendário

3.4 O sistema apresenta uma mensagem a informar do sucesso da operação

3.5 O sistema apresenta o calendário actualizado

4. Caso ocorra algum erro, o mesmo é apresentado ao utilizador