

\chapter{Conclusão}

\paragraph{}

O trabalho apresentado neste relatório, resultou num site onde é possivel gerir momentos de avaliação de cursos leccionados no IPBeja. É possivel um utilizador seguir com o mesmo registo ao longo do tempo, podendo assumir os papéis de Aluno, Docente e/ou Coordenador de Curso.\\
É possivel ao utilizador visualizar as avaliações relacionadas com os papéis por ele desempenhados e, mais especificamente, enquanto Coordenador de Curso: validar ou cancelar a validação de uma avaliação, enquanto Docente: marcar ou cancelar a marcação de uma avaliação e enquanto Aluno: inscrever-se ou cancelar a inscrição num momento de avaliação.\\
Para gerar as página html foi utilizado o motor de templates \code{smarty} que torna mais simples e de melhor compreensão essas tarefas.\\
Por uma questão de segurança, com o intuito de evitar erros de programação ou acessos indevidos, foram criados stored procedures para controlar o acesso ou alterações aos momentos de avaliação.\\
Como trabalho futuro deverá ser implementada uma forma de criação de um login à base de dados para cada utilizador para que se possa limitar o acesso directo às tabelas e o mesmo seja apenas feito através de stored procedures onde será utilizado como referência para as permissões de acesso aos dados o utilizador logado ao invés da disponibilização do id de utilizador como está feito actualmente.