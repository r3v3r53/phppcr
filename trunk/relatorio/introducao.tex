

\chapter{Introdução}

\paragraph{}

\section{Objectivos}
Este relatório insere-se na avaliação da disciplina "Programação Centrada na Rede", leccionada no 2º semestre do 3º ano da licenciatura em Engenharia Informática.\\ 
O trabalho surge na sequência do projecto interdicsiplinar de Engenharia de Software, Bases de Dados I e II, Hipermédia e Acessibilidade e, por último, Programação Centrada na Rede.\\
O tema é o desenvolvimento de uma aplicação que permita gerir os momentos de avaliação das diferentes disciplinas dos cursos da Estig.
Nele pretende-se apresentar um resumo da análise e a fase de desenho do desenvolvimento da aplicação.\\

\section{Problema}
Actualmente, cada docente marca as suas diferentes avaliações de forma individual, podendo ou não enviar essa informação ao Coordenador de Curso. Este, para ter noção da distribuição de carga dos momentos de avaliação ao longo de um semestre tem que, manualmente, registar essa informação, que poderá ser alterada posteriomente, sem que ele receba qualquer notificação. Por outro lado não existe um repositório que centralize todos os momentos de avaliação realizados nas disciplinas ao longo do semestre e que seja transparente para os seus diferentes beneficiários.

\section{Objectivos}
Neste projecto pretende-se a criação de uma aplicação que centralize o registo de todos os momentos de avaliação mantendo os respectivos intervenientes actualizados acerca das alterações às mesmas.\\
Por forma a evitar uma carga excessiva aos alunos, as avaliações terão obrigatóriariamente que ser analisadas pelo coordenador do curso em causa.\\
Na aplicação será também implementada forma para facilitar a percepção do estado das avaliações por parte dos actores.\\


