

\chapter{Conclusão}

\paragraph{}

No trabalho efectuado sobre o tema em na disciplina de Bases de Dados 2, optámos por uma forte política de segurança ao nível da base de dados. Visto que o trabalho actual irá ser implementado em mySQL ao invés de T-SQL, poderão algumas implementações não ser compatíveis com o novo tipo de base de dados. Nomeadamente a criação de um login no servidor e acesso à bd limitando o acesso aos dados de acordo com o perfil. Limitando o acesso directo às tabelas, optando por execução de stored procedures e algumas funcionalidades dos triggers como por exemplo o envio de emails.\\
Nos casos em que não for possivel implementar algumas destas funcionalidades ao nível da base de dados, terão que ser implementadas pelo PHP ou C\# nas respectivas classes.\\
É possivel também que durante a implementação, dadas estas restrições, se chegue à conclusão que é necessário a criação de mais classes representando algumas tabelas da base de dados.\\

No geral com será possível o coordenador de um curso validar, cancelar ou alterar a data de uma avaliação. O docente tem a possibilidade de criar, cancelar ou alterar um momento de avaliação. E o aluno pode Inscrever-se ou cancelar a inscrição numa avaliação.\\
Todos os utilizadores poderão ver os momentos de avaliação relacionados com o seu perfil e, também, alterar os seus dados pessoais, ou recuperar os dados de acesso ao sistema.\\
É possivel um novo utilizador criar um novo registo, no entanto não terá acesso a nenhuma avaliação, visto não estar prevista a implementação da gestão de perfis de utilizadores. Tal funcionalidade terá de ficar para trabalho futuro.